% download perl: 
% https://cli-msi.s3.amazonaws.com/ActivePerl-5.28.msi
% download miktex:
% https://miktex.org/download/ctan/systems/win32/miktex/setup/windows-x64/basic-miktex-21.1-x64.exe
% download latex workskop
% https://marketplace.visualstudio.com/items?itemName=James-Yu.latex-workshop
\documentclass[12pt]{article}

\usepackage[margin=1.0in]{geometry}
\usepackage{amsthm}
\usepackage{amsmath}

\newtheorem{definition}{Definition}

\setlength\parindent{0pt}
\setlength{\parskip}{1em}

\begin{document}

\section{Big-O Notation}

This document contains some basic proofs related to Big-O Notation.

\subsection{Introduction}

\begin{itemize}
\item We use Big-O notation to express the amount of resources used by algorithms (time complexity and space complexity).
\item Denote by $f(N)$ the running time of a program for inputs of size $N$.
\begin{itemize}
\item Ex 1. $f(x) = 1.5N^2 + 4N + 3$
\item Ex 2. $f(x) = 2N^4 + 10N^3 + 4N^2 + 900log(N) + 3000$
\item Ex 3. $f(x) = 2N - 100$ for $(N > 10)$
\end{itemize}
\item The Big-O notiation will be the fastest increasing term for large inputs.
\end{itemize}

For example: if $f(N) = 2N^4 + 10N^3 + 4N^2 + 900log(N) + 3$ then the time complexity is $O(N^4)$.

\subsection{Why are we ignoring low order terms?}

Because they become negligible as $N$ grows.

For example: if $f(N) = 4N^4 + 100N^3 + 9000N^2 + N$.

\begin{center}
\begin{tabular}{| c | c | c | c | c | }
\hline
& $4N^4$ & $100N^3$ & $9000N^2$ & $N$ \\ \hline
$N=100$ & $4 \times 10^8$ & $10^8$ & $9 \times 10^7$ & 100 \\ \hline
$N=10^8$ & $4 \times 10^{32}$ & $10^{26}$ & $9 \times 10^{19}$ & $10^8$ \\ \hline
$N=10^{12}$ & $4 \times 10^{48}$ & $10^{38}$ & $9 \times 10^{27}$ & $10^{12}$ \\ \hline
\end{tabular}
\end{center}

\subsection{Why are we ignoring multiplicative constants?}

Because if we buy a computer that runs twice as fast or a computer with 4 cores, then the running time decreases accordingly.

But it will not change the order of magnitude.

In practice, constants do matter!

In theory, we ignore them.

\subsection{Formal Definition}

\begin{definition}
Let $f(N)$ and $g(N)$ be two functions on positive integers.

We say that $f = O(g)$ if there exists a large constant $C$ (ex. $C = 1000$) such that $f(N) < C*g(N)$ for all sufficiently large $N$.

Equivalently, $f = O(g)$ if there is a $C > 0$ (ex. $C = 1000$) such that $f(N)/g(N) < $ for all $N$ large enough.

\begin{equation}
f = O(g) \hspace{5mm} \textnormal{if} \hspace{5mm} \lim_{\sup N \to \infty} \frac{f(N)}{g(N)} < \infty
\end{equation}

\end{definition}

\subsection{Examples}

Example 1. $f(N) = 5N^2 + 4N + 3$. We want to show $f = O(N^2)$.

\begin{proof}
Let $C$ = 12. Then

\begin{align*}
F(N) &= 5N^2 + 4N + 3 < 5N^2 + 4N \\
&< 5N^2 + 4N^2 + 3N^2 \hspace{5mm} [\textnormal{for $N > 2$}] \\
&= 12N^2 \\
&= CN^2
\end{align*}

Therefore $f = O(N^2)$.

\end{proof}

Example 2. $f(N) = 5N^2 + 4N + 3$. We want to show $f = O(N^3)$.

\begin{proof}
Let $C = 12$. Then

\begin{align*}
f(N) &= 5N^2 + 4N + 3 \\
&< 5N^3 + 4N^3 + 3N^3 \hspace{5mm} [\textnormal{for $N > 2$}] \\
&= 12N^3 \\
&< CN^3
\end{align*}

Therefore $f = O(N^3)$.

\end{proof}

\subsection{Common orders of magnitude}

\begin{itemize}
\item $N^2 = O(N^a)$ for all $a \geq 2$
\item $log(N)$ is smaller than any power of $N$ for all large enough $N$

\begin{itemize}
\item That is, $log(N) = O(N^{0.1})$ or $log^{10}(N) = O(N)$.
\end{itemize}

\end{itemize}

\end{document}