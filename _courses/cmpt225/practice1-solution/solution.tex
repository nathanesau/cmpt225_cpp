\documentclass[12pt]{article}

\usepackage[margin=1.0in]{geometry}
\usepackage[links,assignheader,hideanswers]{uassign}

\lhead{practice1-solution \\ cmpt 225 (Spring 2021)}
\rhead{Nathan Esau \\ 301197568}

\begin{document}

\begin{question}
Sort the functions from lowest asymptotic order to highest asymptotic order. Note: some of them may have the same asymptotic order.

\begin{itemize}
\item $f_{1}(n) = n + n^{3}$
\item $f_{2}(n) = nlog(n)$
\item $f_{3}(n) = log(n)^2$
\item $f_{4}(n) = 1.5^n$
\item $f_{5}(n) = n^{1.5}$
\item $f_{6}(n) = n^{log(n)}$
\item $f_{7}(n) = 2^{n}$
\item $f_{8}(n) = 2^{n-1}$
\item $f_{9}(n) = n!$
\item $f_{10}(n) = n log(n)^2$
\end{itemize}

\end{question}

\begin{solution}
question 1 solution
\end{solution}

\begin{question}
find an asymptotic order of a function $k(n)$ such that $k(n)^{k(n)} = n$.
\end{question}

\begin{solution}
question 2 solution
\end{solution}

\begin{question}
find an asymptotic order of a function $k(n)$ such that $k(n)^{k(n)} = n^{2}$.
\end{question}

\begin{solution}
question 3 solution
\end{solution}

\begin{question}
find an asymptotic order of a function $k(n)$ such that $k(n)^{k(n)} = 2^{n}$.
\end{question}

\begin{solution}
question 4 solution
\end{solution}

\begin{question}
Let $f(n) = 2^{n}$ and $g(n) = 3^{n}$. It is true that $f = \Theta(g)$?
\end{question}

\begin{solution}
question 5 solution
\end{solution}

\begin{question}
Suppose that $f = O(g)$ and $g = O(h)$. Prove formally that $f = O(h)$.
\end{question}

\begin{solution}
question 6 solution
\end{solution}

\begin{question}
Suppose that $f = \Omega(g)$ and $g = \Omega(h)$. Prove formally that $f = \Omega(h)$.
\end{question}

\begin{solution}
question 7 solution
\end{solution}

\begin{question}
Suppose that $f = \Theta(g)$ and $f = \Theta(h)$. Does this imply that $g = \Theta(h)$?
\end{question}

\begin{solution}
question 8 solution
\end{solution}

\begin{question}
It is possible that $f = O(g)$, $g = \Omega(h)$ and $f = \Theta(h)$?
\end{question}

\begin{solution}
question 9 solution
\end{solution}

\begin{question}
It is possible that $f = O(g)$, $g = \Theta(h)$ and $f = O(h)$?
\end{question}

\begin{solution}
question 10 solution
\end{solution}

\begin{question}
It is possible that $f = O(g)$, $g = \Omega(h)$ and $f = \Omega(h)$?
\end{question}

\begin{solution}
question 11 solution
\end{solution}

\end{document}