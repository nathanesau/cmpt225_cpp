\documentclass[12pt]{article}

\usepackage[margin=1.0in]{geometry}
\usepackage[links,assignheader]{uassign}

\newtheorem{theorem}{Theorem}

\lhead{practice2-solution \\ cmpt 225 (Spring 2021)}
\rhead{Nathan Esau \\ 301197568}

\begin{document}

\section*{Background}

\begin{theorem}
Suppose we are given recurrence relation $T(n) = a \times T(n/b) + f(n)$. Let $c = log_{b}(a)$. Then

\begin{enumerate}
\item If $f(n) = O(n^d)$ for $d < c$ then $T(n) = \Theta(n^c)$
\item If $f(n) = \Theta(n^c)$ then $T(n) = \Theta(n^c log(n))$
\item If $f(n) = \Omega(n^c)$ then $T(n) = \Theta(f)$
\end{enumerate}

Note that $\Omega$ is asymptotic lower bound, $O$ is asymptotic upper bound and $\Theta$ is exact bound.

\end{theorem}

\section*{Questions}

\begin{question}
Apply Master Theorem to find the running time of $T(n) = 2T(n/2) + O(1)$.
\end{question}

\begin{solution}
Here $c = log_{2}(2) = 1$ and $d = 0$. Since $d < c$ then $T(n) = \Theta(n)$.
\end{solution}

\begin{question}
Apply Master Theorem to find the running time of $T(n) = 3T(n/2) + O(1)$.
\end{question}

\begin{solution}
Here $c = log_{2}(3) = 1.58$ and $d = 0$. Since $d < c$ then $T(n) = \Theta(n^{1.58})$.
\end{solution}

\begin{question}
Apply Master Theorem to find the running time of $T(n) = 4T(n/2) + O(n)$.
\end{question}

\begin{solution}
Here $c = log_{2}(4) = 2$ and $d = 1$. Since $d < c$ then $T(n) = \Theta(n^{2})$.
\end{solution}

\begin{question}
Apply Master Theorem to find the running time of $T(n) = 3T(n/2) + O(n^2)$.
\end{question}

\begin{solution}
Here $c = log_{2}(3) = 1.58$ and $d = 2$. We have case three, so $T(n) = O(n^2)$.
\end{solution}

\begin{question}
Apply Master Theorem to find the running time of $T(n) = 3T(n/3) + O(n)$.
\end{question}

\begin{solution}
Here $c = log_{3}(3) = 1$ and $d = 1$. We have case two, so $T(n) = O(n log(n))$.
\end{solution}

\begin{question}
Apply Master Theorem to find the running time of $T(n) = 8T(n/2) + O(n^2)$.
\end{question}

\begin{solution}
Here $c = log_{8}(2) = 3$ and $d = 2$. We have case one, so $T(n) = O(n^3)$.
\end{solution}

\begin{question}
Apply Master Theorem to find the running time of $T(n) = 8T(n/2) + O(n^3)$.
\end{question}

\begin{solution}
Here $c = log_{8}(2) = 3$ and $d = 3$. We have case two, so $T(n) = O(n^3 log(n))$.
\end{solution}

\begin{question}
Apply Master Theorem to find the running time of $T(n) = 8T(n/2) + O(n^7)$.
\end{question}

\begin{solution}
Here $c = log_{8}(2) = 3$ and $d = 7$. We have case three, so $T(n) = O(n^7)$.
\end{solution}

\begin{question}
Apply Master Theorem to find the running time of $T(n) = 5T(n/3) + O(n^{1.5})$.
\end{question}

\begin{solution}
Here $c = log_{5}(3) = 1.46$ and $d = 1.5$. We have case three, so $T(n) = O(n^{1.5})$.
\end{solution}

\begin{question}
Apply Master Theorem to find the running time of $T(n) = 5T(n/3) + O(n^{1.4})$.
\end{question}

\begin{solution}
Here $c = log_{5}(3) = 1.46$ and $d = 1.4$. We have case one, so $T(n) = O(n^{1.46})$.
\end{solution}

\begin{question}
Apply Master Theorem to find the running time of $T(n) = 3T(n/3) + O(n)$.
\end{question}

\begin{solution}
Here $c = log_{3}(3) = 1$ and $d = 1$. We have case two, so $T(n) = O(nlog(n))$.
\end{solution}

\begin{question}
Apply Master Theorem to find the running time of $T(n) = 9T(n/3) + O(n^2)$.
\end{question}

\begin{solution}
Here $c = log_{3}(9) = 2$ and $d = 2$. We have case two, so $T(n) = O(n^2log(n))$.
\end{solution}

\end{document}